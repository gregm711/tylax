\documentclass[11pt,letterpaper]{article}
\usepackage[left=1in,right=1in,top=0.75in,bottom=0.75in]{geometry}
\usepackage{amsmath,amssymb}
\usepackage{graphicx}
\usepackage{hyperref}
\usepackage[table]{xcolor}
\usepackage{booktabs}
\usepackage{enumitem}
\usepackage{multirow}
\usepackage{multicol}
\usepackage{array}
\usepackage{textcomp}
\usepackage{float}
\usepackage{amsthm}
\usepackage{titlesec}
\titlespacing*{\section}{0pt}{1em}{0.5em}
\titlespacing*{\subsection}{0pt}{0.8em}{0.3em}
\titlespacing*{\subsubsection}{0pt}{0.6em}{0.2em}
\titlespacing*{\paragraph}{0pt}{0.5em}{0.1em}
\usepackage{newcomputermodern}
\theoremstyle{plain}
\newtheorem{theorem}{Theorem}[section]
\newtheorem{lemma}[theorem]{Lemma}
\newtheorem{corollary}[theorem]{Corollary}
\newtheorem{proposition}[theorem]{Proposition}
\newtheorem{claim}[theorem]{Claim}
\newtheorem{axiom}[theorem]{Axiom}
\theoremstyle{definition}
\newtheorem{definition}[theorem]{Definition}
\newtheorem{example}[theorem]{Example}
\theoremstyle{remark}
\newtheorem{remark}[theorem]{Remark}
\usepackage{setspace}
\setstretch{1.33}
\providecommand{\textsubscript}[1]{$_{\text{#1}}$}
\begin{document}

\begin{center}
{\LARGE \textbf{ MATH 301: Real Analysis }}
\par\vspace{0.3em}
{\large Lecture Notes - Week 5 }
\par\vspace{0.3em}
\textcolor[HTML]{666666}{ Prof. Jane Smith · Fall 2024 · Your University }
\par\vspace{0.3em}
{\small \textit{ Notes by: Your Name · Last updated: "[month repr:long] [day], [year]" }}
\end{center}

\par\vspace{1em}

\par\vspace{1em}

\section{Sequences and Convergence}

\subsection{Introduction}

In this lecture, we study the fundamental concept of convergence for sequences of real numbers. Understanding sequences is essential for defining limits, continuity, and the foundations of calculus.

\{ theorem-counter.step() block( width: 100\%, inset: 12pt, fill: rgb(""), stroke: (left: 3pt + rgb("")), radius: (right: 4pt), )[ \textbf{Definition } (Sequence) \textbf{.}

A \textbf{sequence} of real numbers is a function $a: \mathbb{N} \to \mathbb{R}$. We typically write $a_n$ for $a(n)$ and denote the sequence as $(a_n)_(n=1)^infinity$ or simply $(a_n)$.

] \}

\{ block( width: 100\%, inset: 12pt, fill: rgb(""), stroke: (left: 3pt + rgb("")), radius: (right: 4pt), )[ \textbf{Example.}

Consider the sequence $a_n = 1/n$. The first few terms are: $a_1 = 1, quad a_2 = 1/2, quad a_3 = 1/3, quad a_4 = 1/4, quad \ldots$ Intuitively, this sequence "approaches" 0 as $n$ gets large.

] \}

\subsection{Convergence of Sequences}

\{ theorem-counter.step() block( width: 100\%, inset: 12pt, fill: rgb(""), stroke: (left: 3pt + rgb("")), radius: (right: 4pt), )[ \textbf{Definition } (Convergence) \textbf{.}

A sequence $(a_n)$ \textbf{converges} to a limit $L \in \mathbb{R}$ if for every $\epsilon > 0$, there exists $N \in \mathbb{N}$ such that for all $n > N$: $|a_n - L| < \epsilon$ We write $\lim_(n \to infinity) a_n = L$ or $a_n \to L$.

] \}

\{ block( width: 100\%, inset: 12pt, fill: rgb(""), stroke: (left: 3pt + rgb("")), radius: (right: 4pt), )[ \textbf{Remark.}

The definition says that $a_n$ can be made arbitrarily close to $L$ by taking $n$ sufficiently large. The key insight is that $N$ depends on the choice of $\epsilon$: smaller $\epsilon$ typically requires larger $N$.

] \}

\{ block( width: 100\%, inset: 12pt, fill: rgb(""), stroke: (left: 3pt + rgb("")), radius: (right: 4pt), )[ \textbf{Example.}

Let's prove that $\lim_(n \to infinity) 1/n = 0$.

\textbf{Proof.} Let $\epsilon > 0$ be given. We need to find $N$ such that $|1/n - 0| < \epsilon$ for all $n > N$.

Choose $N = \left\lceil 1/\epsilon \right\rceil$. Then for $n > N$: $|1/n - 0| = 1/n < 1/N \le \epsilon$

Thus $1/n \to 0$. $square$

] \}

\subsection{Properties of Limits}

\{ theorem-counter.step() block( width: 100\%, inset: 12pt, fill: rgb(""), stroke: (left: 3pt + rgb("")), radius: (right: 4pt), )[ \textbf{Theorem } (Uniqueness of Limits) \textbf{.}

If $(a_n)$ converges, then its limit is unique.

] \}

\{ block( width: 100\%, inset: (x: 12pt, y: 8pt), )[ \textit{Proof.}

Suppose $a_n \to L$ and $a_n \to M$ with $L eq.not M$. Let $\epsilon = |L - M|/2 > 0$.

By convergence to $L$, there exists $N_1$ such that $|a_n - L| < \epsilon$ for $n > N_1$.

By convergence to $M$, there exists $N_2$ such that $|a_n - M| < \epsilon$ for $n > N_2$.

For $n > \max\left(N_1, N_2\right)$, by the triangle inequality: $|L - M| \le |L - a_n| + |a_n - M| < \epsilon + \epsilon = |L - M|$

This is a contradiction, so $L = M$.

$square$ ] \}

\{ theorem-counter.step() block( width: 100\%, inset: 12pt, fill: rgb(""), stroke: (left: 3pt + rgb("")), radius: (right: 4pt), )[ \textbf{Theorem } (Algebraic Limit Theorem) \textbf{.}

Let $(a_n) \to L$ and $(b_n) \to M$. Then:

\begin{enumerate}
  \item  \begin{equation}
(a_n + b_n) \to L + M
\end{equation}
  \item $(c dot a_n) \to c dot L$ for any constant $c \in \mathbb{R}$
  \item  \begin{equation}
(a_n dot b_n) \to L dot M
\end{equation}
  \item $(a_n / b_n) \to L / M$ provided $M eq.not 0$ and $b_n eq.not 0$ for all $n$
\end{enumerate}

] \}

\{ block( width: 100\%, inset: (x: 12pt, y: 8pt), )[ \textit{Proof.}

We prove part (1). Let $\epsilon > 0$. Since $a_n \to L$, there exists $N_1$ such that $|a_n - L| < \epsilon/2$ for $n > N_1$. Similarly, there exists $N_2$ such that $|b_n - M| < \epsilon/2$ for $n > N_2$.

For $n > N = \max\left(N_1, N_2\right)$:

\begin{align}
|(a_n + b_n) - (L + M)| &\le |a_n - L| + |b_n - M| \
                            &< \epsilon/2 + \epsilon/2 = \epsilon
\end{align}

Parts (2)-(4) are left as exercises.

$square$ ] \}

\subsection{Bounded Sequences}

\{ theorem-counter.step() block( width: 100\%, inset: 12pt, fill: rgb(""), stroke: (left: 3pt + rgb("")), radius: (right: 4pt), )[ \textbf{Definition } (Bounded Sequence) \textbf{.}

A sequence $(a_n)$ is \textbf{bounded} if there exists $M > 0$ such that $|a_n| \le M$ for all $n \in \mathbb{N}$.

] \}

\{ theorem-counter.step() block( width: 100\%, inset: 12pt, fill: rgb(""), stroke: (left: 3pt + rgb("")), radius: (right: 4pt), )[ \textbf{Lemma } ( Every convergent sequence is bounded. ) \textbf{.}

] \}

\{ block( width: 100\%, inset: (x: 12pt, y: 8pt), )[ \textit{Proof.}

Let $a_n \to L$. Taking $\epsilon = 1$, there exists $N$ such that $|a_n - L| < 1$ for all $n > N$.

For $n > N$: $|a_n| \le |a_n - L| + |L| < 1 + |L|$.

Let $M = \max{|a_1|, |a_2|, \ldots, |a_N|, 1 + |L|}$.

Then $|a_n| \le M$ for all $n \in \mathbb{N}$.

$square$ ] \}

\{ block( width: 100\%, inset: 12pt, fill: rgb(""), stroke: (left: 3pt + rgb("")), radius: (right: 4pt), )[ \textbf{Remark.}

The converse is false! The sequence $a_n = (-1)^n$ is bounded (by $M = 1$) but does not converge.

] \}

\section{Monotone Sequences}

\{ theorem-counter.step() block( width: 100\%, inset: 12pt, fill: rgb(""), stroke: (left: 3pt + rgb("")), radius: (right: 4pt), )[ \textbf{Definition } (Monotone Sequence) \textbf{.}

A sequence $(a_n)$ is:

\begin{itemize}
  \item \textbf{Increasing} if $a_n \le a_(n+1)$ for all $n$
  \item \textbf{Decreasing} if $a_n \ge a_(n+1)$ for all $n$
  \item \textbf{Monotone} if it is either increasing or decreasing
\end{itemize}

] \}

\{ theorem-counter.step() block( width: 100\%, inset: 12pt, fill: rgb(""), stroke: (left: 3pt + rgb("")), radius: (right: 4pt), )[ \textbf{Theorem } (Monotone Convergence Theorem) \textbf{.}

Every bounded monotone sequence converges.

] \}

This is one of the most important theorems in real analysis, relying on the completeness of $\mathbb{R}$.

\{ block( width: 100\%, inset: 12pt, fill: rgb(""), stroke: (left: 3pt + rgb("")), radius: (right: 4pt), )[ \textbf{Example.}

Consider the sequence defined recursively by: $a_1 = 1, quad a_(n+1) = 1/2 (a_n + 2/a_n)$

This is the Babylonian method for computing $\sqrt{2}$. One can show:

\begin{enumerate}
  \item $(a_n)$ is decreasing for $n \ge 2$
  \item $(a_n)$ is bounded below by $\sqrt{2}$
\end{enumerate}

By the Monotone Convergence Theorem, $(a_n)$ converges. Taking the limit of both sides of the recurrence shows $L = \sqrt{2}$.

] \}

\par\vspace{1em}

\par\vspace{0.5em}

\begin{center}
{\small \textcolor[HTML]{666666}{ \textbf{Exercises:} 3.1, 3.2, 3.5, 3.8 from the textbook · \textbf{Reading:} Chapter 3.1-3.3 }}
\end{center}
\end{document}
