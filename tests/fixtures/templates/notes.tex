\documentclass[11pt,letterpaper]{article}
\usepackage[left=1in,right=1in,top=0.75in,bottom=0.75in]{geometry}
\usepackage{amsmath,amssymb}
\usepackage{graphicx}
\usepackage{hyperref}
\usepackage[table]{xcolor}
\usepackage{enumitem}
\usepackage{multirow}
\usepackage{multicol}
\usepackage{array}
\usepackage{textcomp}
\usepackage{amsthm}
\usepackage{newcomputermodern}
\theoremstyle{plain}
\newtheorem{theorem}{Theorem}[section]
\newtheorem{lemma}[theorem]{Lemma}
\newtheorem{corollary}[theorem]{Corollary}
\newtheorem{proposition}[theorem]{Proposition}
\newtheorem{claim}[theorem]{Claim}
\newtheorem{axiom}[theorem]{Axiom}
\theoremstyle{definition}
\newtheorem{definition}[theorem]{Definition}
\newtheorem{example}[theorem]{Example}
\theoremstyle{remark}
\newtheorem{remark}[theorem]{Remark}
\usepackage{setspace}
\setstretch{1.76}
\providecommand{\textsubscript}[1]{$_{\text{#1}}$}
\begin{document}

\begin{center}
\textbf{ MATH 301: Real Analysis }\\ Lecture Notes - Week 5 \\\textcolor[HTML]{666666}{ Prof. Jane Smith · Fall 2024 · Your University }\\\textit{ Notes by: Your Name · Last updated: "[month repr:long] [day], [year]" }
\end{center}

\par\vspace{1em}

\par\vspace{1em}

\section{Sequences and Convergence}

\subsection{Introduction}

In this lecture, we study the fundamental concept of convergence for sequences of real numbers. Understanding sequences is essential for defining limits, continuity, and the foundations of calculus.

Sequence Sequence A \textbf{sequence} of real numbers is a function

\begin{equation}
a: \mathbb{N} \to \mathbb{R}
\end{equation}

. We typically write

\begin{equation}
a_n
\end{equation}

for

\begin{equation}
a(n)
\end{equation}

and denote the sequence as

\begin{equation}
(a_n)_(n=1)^infinity
\end{equation}

or simply

\begin{equation}
(a_n)
\end{equation}

.

Consider the sequence

\begin{equation}
a_n = 1/n
\end{equation}

. The first few terms are:

\begin{equation}
a_1 = 1, quad a_2 = 1/2, quad a_3 = 1/3, quad a_4 = 1/4, quad \ldots
\end{equation}

Intuitively, this sequence "approaches" 0 as

\begin{equation}
n
\end{equation}

gets large.

\subsection{Convergence of Sequences}

Convergence Convergence A sequence

\begin{equation}
(a_n)
\end{equation}

\textbf{converges} to a limit

\begin{equation}
L \in \mathbb{R}
\end{equation}

if for every

\begin{equation}
\epsilon > 0
\end{equation}

, there exists

\begin{equation}
N \in \mathbb{N}
\end{equation}

such that for all

\begin{equation}
n > N
\end{equation}

:

\begin{equation}
|a_n - L| < \epsilon
\end{equation}

We write

\begin{equation}
\lim_(n \to infinity) a_n = L
\end{equation}

or

\begin{equation}
a_n \to L
\end{equation}

.

The definition says that

\begin{equation}
a_n
\end{equation}

can be made arbitrarily close to

\begin{equation}
L
\end{equation}

by taking

\begin{equation}
n
\end{equation}

sufficiently large. The key insight is that

\begin{equation}
N
\end{equation}

depends on the choice of

\begin{equation}
\epsilon
\end{equation}

: smaller

\begin{equation}
\epsilon
\end{equation}

typically requires larger

\begin{equation}
N
\end{equation}

.

Let's prove that

\begin{equation}
\lim_(n \to infinity) 1/n = 0
\end{equation}

.

\textbf{Proof.} Let

\begin{equation}
\epsilon > 0
\end{equation}

be given. We need to find

\begin{equation}
N
\end{equation}

such that

\begin{equation}
|1/n - 0| < \epsilon
\end{equation}

for all

\begin{equation}
n > N
\end{equation}

.

Choose

\begin{equation}
N = \left\lceil 1/\epsilon \right\rceil
\end{equation}

. Then for

\begin{equation}
n > N
\end{equation}

:

\begin{equation}
|1/n - 0| = 1/n < 1/N \le \epsilon
\end{equation}

Thus

\begin{equation}
1/n \to 0
\end{equation}

.

\begin{equation}
square
\end{equation}

\subsection{Properties of Limits}

Uniqueness of Limits Uniqueness of Limits If

\begin{equation}
(a_n)
\end{equation}

converges, then its limit is unique.

Suppose

\begin{equation}
a_n \to L
\end{equation}

and

\begin{equation}
a_n \to M
\end{equation}

with

\begin{equation}
L eq.not M
\end{equation}

. Let

\begin{equation}
\epsilon = |L - M|/2 > 0
\end{equation}

.

By convergence to

\begin{equation}
L
\end{equation}

, there exists

\begin{equation}
N_1
\end{equation}

such that

\begin{equation}
|a_n - L| < \epsilon
\end{equation}

for

\begin{equation}
n > N_1
\end{equation}

.

By convergence to

\begin{equation}
M
\end{equation}

, there exists

\begin{equation}
N_2
\end{equation}

such that

\begin{equation}
|a_n - M| < \epsilon
\end{equation}

for

\begin{equation}
n > N_2
\end{equation}

.

For

\begin{equation}
n > \max\left(N_1, N_2\right)
\end{equation}

, by the triangle inequality:

\begin{equation}
|L - M| \le |L - a_n| + |a_n - M| < \epsilon + \epsilon = |L - M|
\end{equation}

This is a contradiction, so

\begin{equation}
L = M
\end{equation}

.

Algebraic Limit Theorem Algebraic Limit Theorem Let

\begin{equation}
(a_n) \to L
\end{equation}

and

\begin{equation}
(b_n) \to M
\end{equation}

. Then:

\begin{enumerate}
  \item $(a_n + b_n) \to L + M$
  \item $(c dot a_n) \to c dot L$ for any constant $c \in \mathbb{R}$
  \item $(a_n dot b_n) \to L dot M$
  \item $(a_n / b_n) \to L / M$ provided $M eq.not 0$ and $b_n eq.not 0$ for all $n$
\end{enumerate}

We prove part (1). Let

\begin{equation}
\epsilon > 0
\end{equation}

. Since

\begin{equation}
a_n \to L
\end{equation}

, there exists

\begin{equation}
N_1
\end{equation}

such that

\begin{equation}
|a_n - L| < \epsilon/2
\end{equation}

for

\begin{equation}
n > N_1
\end{equation}

. Similarly, there exists

\begin{equation}
N_2
\end{equation}

such that

\begin{equation}
|b_n - M| < \epsilon/2
\end{equation}

for

\begin{equation}
n > N_2
\end{equation}

.

For

\begin{equation}
n > N = \max\left(N_1, N_2\right)
\end{equation}

:

\begin{align}
|(a_n + b_n) - (L + M)| &\le |a_n - L| + |b_n - M| \
                            &< \epsilon/2 + \epsilon/2 = \epsilon
\end{align}

Parts (2)-(4) are left as exercises.

\subsection{Bounded Sequences}

Bounded Sequence Bounded Sequence A sequence

\begin{equation}
(a_n)
\end{equation}

is \textbf{bounded} if there exists

\begin{equation}
M > 0
\end{equation}

such that

\begin{equation}
|a_n| \le M
\end{equation}

for all

\begin{equation}
n \in \mathbb{N}
\end{equation}

.

Every convergent sequence is bounded.

Every convergent sequence is bounded.

Let

\begin{equation}
a_n \to L
\end{equation}

. Taking

\begin{equation}
\epsilon = 1
\end{equation}

, there exists

\begin{equation}
N
\end{equation}

such that

\begin{equation}
|a_n - L| < 1
\end{equation}

for all

\begin{equation}
n > N
\end{equation}

.

For

\begin{equation}
n > N
\end{equation}

:

\begin{equation}
|a_n| \le |a_n - L| + |L| < 1 + |L|
\end{equation}

.

Let

\begin{equation}
M = \max{|a_1|, |a_2|, \ldots, |a_N|, 1 + |L|}
\end{equation}

.

Then

\begin{equation}
|a_n| \le M
\end{equation}

for all

\begin{equation}
n \in \mathbb{N}
\end{equation}

.

The converse is false! The sequence

\begin{equation}
a_n = (-1)^n
\end{equation}

is bounded (by

\begin{equation}
M = 1
\end{equation}

) but does not converge.

\section{Monotone Sequences}

Monotone Sequence Monotone Sequence A sequence

\begin{equation}
(a_n)
\end{equation}

is:

\begin{itemize}
  \item \textbf{Increasing} if $a_n \le a_(n+1)$ for all $n$
  \item \textbf{Decreasing} if $a_n \ge a_(n+1)$ for all $n$
  \item \textbf{Monotone} if it is either increasing or decreasing
\end{itemize}

Monotone Convergence Theorem Monotone Convergence Theorem Every bounded monotone sequence converges.

This is one of the most important theorems in real analysis, relying on the completeness of

\begin{equation}
\mathbb{R}
\end{equation}

.

Consider the sequence defined recursively by:

\begin{equation}
a_1 = 1, quad a_(n+1) = 1/2 (a_n + 2/a_n)
\end{equation}

This is the Babylonian method for computing

\begin{equation}
\sqrt{2}
\end{equation}

. One can show:

\begin{enumerate}
  \item $(a_n)$ is decreasing for $n \ge 2$
  \item $(a_n)$ is bounded below by $\sqrt{2}$
\end{enumerate}

By the Monotone Convergence Theorem,

\begin{equation}
(a_n)
\end{equation}

converges. Taking the limit of both sides of the recurrence shows

\begin{equation}
L = \sqrt{2}
\end{equation}

.

\par\vspace{1em}

\par\vspace{0.5em}

\begin{center}
\textcolor[HTML]{666666}{ \textbf{Exercises:} 3.1, 3.2, 3.5, 3.8 from the textbook · \textbf{Reading:} Chapter 3.1-3.3 }
\end{center}
\end{document}
