\documentclass[10pt,a4paper]{article}
\usepackage{amsmath,amssymb}
\usepackage{graphicx}
\usepackage{hyperref}
\usepackage[table]{xcolor}
\usepackage{booktabs}
\usepackage{enumitem}
\usepackage{multirow}
\usepackage{multicol}
\usepackage{array}
\usepackage{textcomp}
\usepackage{float}
\usepackage{titlesec}
\titlespacing*{\section}{0pt}{1em}{0.5em}
\titlespacing*{\subsection}{0pt}{0.8em}{0.3em}
\titlespacing*{\subsubsection}{0pt}{0.6em}{0.2em}
\titlespacing*{\paragraph}{0pt}{0.5em}{0.1em}
\providecommand{\textsubscript}[1]{$_{\text{#1}}$}
\begin{document}

\begin{center}
{\huge \textbf{The Atomic Theory as Applied To Gases, with Some Experiments on the Viscosity of Air}}
\par\vspace{1em}
{\large Silas W. Holman Department of Physics}
\par\vspace{0.5em}

\end{center}

\textbf{Thesis Metadata}
\par\vspace{0.5em}
 \begin{itemize}
  \item \textbf{Degree:} Bachelor of Science in Physics (Department of Physics)
  \item \textbf{Supervisor:} Edward C. Pickering, Professor of Physics, Department of Physics
  \item \textbf{Acceptor:} Tertius Castor, Professor of Log Dams, Graduate Officer, Department of Research
  \item \textbf{Degree date:} June 1876
  \item \textbf{Thesis date:} May 18, 1876
  \item \textbf{Reader:} Marcus Gavius Apicius, Professor of Cooking Arts, Department of Food Science
  \item \textbf{Reader:} Marie-Antoine Carême, Professor of Haute Cuisine, Department of Food Science
  \item \textbf{Reader:} Miles Gloriosus, Professor of Personal Pronouns, Department of Rhetoric
\end{itemize}

 \begin{center}
\textbf{Abstract}
\end{center} The developments of the ``kinetic theory'' of gases made within the last ten years have enabled it to account satisfactorily for many of the laws of gases. The mathematical deductions of Clausius, Maxwell and others, based upon the hypothesis of a gas composed of molecules acting upon each other at impact like perfectly elastic spheres, have furnished expressions for the laws of its elasticity, viscosity, conductivity for heat, diffusive power and other properties. For some of these laws we have experimental data of value in testing the validity of these deductions and assumptions. Next to the elasticity, perhaps the phenomena of the viscosity of gases are best adapted to investigation.\footnote{Text from Holman (1876): \href{https://doi.org/10.2307/25138434}{10.2307/25138434}.}

\section{Acknowledgments}

Write your acknowledgments here.

\section{Biographical Sketch}

Silas Whitcomb Holman was born in Harvard, Massachusetts on January 20, 1856. He received his S.B. degree in Physics from MIT in 1876, and then joined the MIT Department of Physics as an Assistant. He became Instructor in Physics in 1880, Assistant Professor in 1882, Associate Professor in 1885, and Full Professor in 1893. Throughout this period, he struggled with increasingly severe rheumatoid arthritis. At length, he was defeated, becoming Professor Emeritus in 1897 and dying on April 1, 1900.

Holman's light burned brilliantly before his tragic and untimely death. He published extensively in thermal physics, and authored textbooks on precision measurement, fundamental mechanics, and other subjects. He established the original Heat Measurements Laboratory. Holman was a much admired teacher among both his students and his colleagues. The reports of his department and of the Institute itself refer to him frequently in the 1880's and 1890's, in tones that gradually shift from the greatest respect to the deepest sympathy.

Holman was a student of Professor Edward C. Pickering, then head of the Physics department. Holman himself became second in command of Physics, under Professor Charles R. Cross, some years later. Among Holman's students, several went on to distinguish themselves, including: the astronomer George E. Hale ('90) who organized the Yerkes and Mt. Wilson observatories and who designed the 200 inch telescope on Mt. Palomar; Charles G. Abbot ('94), also an astrophysicist and later Secretary of the Smithsonian Institution; and George K. Burgess ('96), later Director of the Bureau of Standards.

\section{Introduction}

1-2 Postremo aliquos futuros suspicor, qui me ad alias litteras vocent, genus hoc scribendi, etsi sit elegans, personae tamen et dignitatis esse negent \cite{DKE1969} \cite{ww1920} \cite{kirk2288a} \cite{churchill1948} \cite{gibbs1863}.

\subsection{}

A section discussing the first issue: $J /\psi$A section discussing the first issue: $class("normal", \mathbf{J} /\mathbf{\psi}) class("normal", J /\psi)$

We begin with some ideas from the literature \cite{Fong2015} \cite{sharpe1}.

\[
\frac{\partial}{\partial t}[ \rho e + bar.v arrow(u) bar.v^(2)/ 2 ] + \nabla dot[ \rho h + bar.v arrow(u) bar.v^(2)/ 2 arrow(u) ]
 = zws - \nabla dot arrow(q) + \rho arrow(u) dot arrow(g) + \frac{\partial}{\partial x_(j)}d_(j i)u_(i)
\]

3

4 And more citations \cite{sharpe1} \cite{GSL}. Then we write some more and include our citations \cite{Swaminathan2017IDABRO} \cite{dlmf} \cite{amsmath}. The configuration is shown in Fig. @fig-golden2.

\begin{figure}[H]
\centering

\caption{A figure with two subfigures: (a) first subfigure; (b) second subfigure.\{fig:4\}}
\end{figure}

4

\subsubsection{}

Subsection eqn. (@eqn:WT1)Subsection eqn. (\cite{eqn-WT1}) 5-6

\paragraph{A subsubsection}

7

\[
L(\mathrm{bold(A)} A) = \begin{pmatrix}display(\frac{\phi}{(\phi_(1),\epsilon.alt_(1))}) & 0 & \ldots & \ldots & \ldots & 0 \\ [4 ]
display(\frac{\phi k_(2,1)}{(\phi_(2),\epsilon.alt_(1))}) & display(\frac{\phi}{(\phi_(2),\epsilon.alt_(2))}) & 0 & \ldots & \ldots & 0 \\ [4 ]
display(\frac{\phi k_(3,1)}{(\phi_(3),\epsilon.alt_(1))}) & display(\frac{\phi k_(3,2)}{(\phi_(3),\epsilon.alt_(2))}) & display(\frac{\phi}{(\phi_(3),\epsilon.alt_(3))}) & 0 & \ldots & 0 \\ [4 ]
dots.v & dots.v & "" zws & dots.down & "" zws & dots.v \\ []
display(\frac{\phi k_(n - 1, 1)}{(\phi_(n - 1),\epsilon.alt_(1))}) & display(\frac{\phi k_(n - 1, 2)}{(\phi_(n - 1),\epsilon.alt_(2))}) & \ldots & display(\frac{\phi k_(n - 1,n - 2)}{(\phi_(n - 1),\epsilon.alt_(n - 2))}) & display(\frac{\phi}{(\phi_(n - 1),\epsilon.alt_(n - 1))}) & 0 \\ [4 ]
display(\frac{\phi k_(n,1)}{(\phi_(n),\epsilon.alt_(1))}) & display(\frac{\phi k_(n,2)}{(\phi_(n),\epsilon.alt_(2))}) & \ldots & \ldots & display(\frac{\phi k_(n,n - 1)}{(\phi_(n),\epsilon.alt_(n - 1))}) & display(\frac{\phi}{(\phi_(n),\epsilon.alt_(n))})\end{pmatrix}
\]

\subsection{Description our paradigm}
\label{ch1-theidea}

8 No dissertation is complete without footnotes.\footnote{First footnote. $a_(h) = F_(m)$ See section \cite{sec-stratified-flow}.}\footnote{Another interesting detail.}\footnote{And another really important idea to have in mind \cite{reynolds1958} \cite{clauser56} \cite{lienhard2020} \cite{johnson1980} \cite{johnson1965} \cite{mpl}.}

\begin{figure}[H]
\centering
\includegraphics{missing-image.pdf}
\label{example-image-b}
\end{figure}

\subsubsection{Conversion to a metaheuristic}

11-12 This concept is discussed further in section \cite{sec-stratified-flow}, and Refs. \cite{euler1740} \cite{fourier1822}.

\subsection{Other generalizations}

\subsubsection{The most general case}

7 And another citation, so that our sources will be unambiguous \cite{montijano2014}.

\[
"x Na(NH4)HPO4 \to (NaPO3)_x + x NH3^ + x H2O" \ [0 . 5 e m ]
"^234_90Th \to^0_-1$\beta${} +^234_91Pa" \ [0 . 5 e m ]
"SO4^2- + Ba^2+ \to BaSO4 v" \ [0 . 5 e m ]
"Zn^2+
\Leftrightarrow+ 2OH-+ 2H+
$limits("Zn(OH)2 v")_("amphoteric hydroxide")$
\Leftrightarrow+ 2OH-+ 2H+
$limits("Zn(OH)4]^2-")_("tetrahydroxozincate")$"
\]

These examples of chemical formulæ are copied directly from the documentation of the mhchem package, which was used to typeset them.

\subsection{Baroclinic generation of vorticity}
\label{sec-stratified-flow}

Substitution of the particle acceleration and application Stokes theorem leads to the \textit{Kelvin-Bjerknes circulation theorem}, for $\rho \ne "fn"(p)$:

\begin{align*}
\frac{d \Gamma}{d t} & zws = \frac{d}{d t} \int_(\mathcal{C}) \mathrm{bold(u)} dot d upright(\mathbf{r}) \
				 & zws = \int_(\mathcal{C}) \frac{D upright(\mathbf{u})}{D t} dot d upright(\mathbf{r}) + underbrace(\int_(\mathcal{C}) \mathrm{bold(u)} dot d frac(d upright(\mathbf{r}), d t))_(= 0) \[- 2 p t ]
 & zws = \int.double_(\mathcal{S}) \nabla times frac(D upright(\mathbf{u}), D t) dot d upright(\mathbf{A}) \
 & zws = \int.double_(\mathcal{S}) \nabla p times \nabla(1/\rho) dot d upright(\mathbf{A})
\end{align*}

Baroclinic generation of vorticity accounts for the sea breeze and various other atmospheric currents in which temperature, rather than pressure, creates density gradients. Further, this phenomenon accounts for ocean currents in straits joining more and less saline seas, with surface currents flowing from the fresher to the saltier water and with bottom current going oppositely.

\begin{figure}[H]
\centering

\caption{The error function and complementary error function\{tab:1\}}
\end{figure}

\subsection{Summary}

12-13

\section{Nomenclature}

\textbf{Roman letters}

\begin{itemize}
  \item \textbf{$\mathcal{C}$} — material curve
  \item \textbf{$\mathrm{bold(r)}$} — material position m
  \item \textbf{$\mathrm{bold(u)}$} — velocity m $"s"^(- 1)$
\end{itemize}

\textbf{Greek letters}

\begin{itemize}
  \item \textbf{$\Gamma$} — circulation $["m"^(2)$ $"s"^(- 1)]$
  \item \textbf{$\rho$} — mass density kg $"m"^(- 3)$
  \item \textbf{$class("normal", \mathbf{\omega}) class("normal", \omega)$} — vorticity $["s"^(- 1)]$
\end{itemize}

\section{Code listing}

This example uses the listings package.

mystyle backgroundcolor=, commentstyle=, numberstyle=, stringstyle=, basicstyle=, breakatwhitespace=false, breaklines=true, numbers=left, numbersep=5pt, showspaces=false, showstringspaces=false, showtabs=false, tabsize=2

language=5.3Lua,style=mystyle

luacode*function print\_rate(kappa,xMin,xMax,npoints,option) local c = 1-kappa*kappa local croot = (1-kappa*kappa)\textasciicircum{}(1/2) local logx = math.log(xMin) local psi = 0 local xstep = (math.log(xMax)-math.log(xMin))/(npoints-1) arg0 = math.sqrt(xMin/c) psi0 = (1/c)*math.exp((kappa*arg0)\textasciicircum{}2)*(erfc(kappa*arg0)-erfc(arg0)) if option\textasciitilde{}=[[]] then tex.sprint("\textbackslash{}\textbackslash{}addplot+["..option.."] coordinates\{") -- addplot+ for color cycle to work else tex.sprint("\textbackslash{}\textbackslash{}addplot+ coordinates\{") end tex.sprint("("..xMin..","..psi0..")") for i=1, (npoints-1) do x = math.exp(logx + xstep) arg = math.sqrt(x/c) karg = kappa*arg if karg<5 then -- this break compensates for exp(karg\textasciicircum{}2), which multiplies the error in the erf approximation... logpsi = -math.log(croot) + karg\textasciicircum{}2 + math.log(erfc(karg)-erfc(arg)) psi = math.exp(logpsi) else psi = (1/(karg) - 1/(2*(karg\textasciicircum{}3)) + 3/(4*(arg\textasciicircum{}5)) )/(1.77245385*croot) -- this is the large x asymptote of the reaction rate end logx = math.log(x) tex.sprint("("..x..","..psi..")") end tex.sprint("\}")endluacode*lstlisting\% listings example (not compatible with tagged pdf, as of Oct. 2025)\%\% MIT Thesis class sample appendix with a long table\%\% version 1.04, 2025/11/02\textbackslash{}chapter\{One-term coefficients for heat conduction\}\textbackslash{}section\{A multipage table of numbers\}This example uses the \textbackslash{}texttt\{longtable\} package: \$\textbackslash{}theta = A\_1 f\_1 \textbackslash{}exp(-\textbackslash{}lambda\_1\textasciicircum{}2\textbackslash{}mkern2mu\textbackslash{}mathrm\{Fo\})\$, \$\textbackslash{}overline\{\textbackslash{}theta\} = D\_1 \textbackslash{}exp(-\textbackslash{}lambda\_1\textasciicircum{}2\textbackslash{}mkern2mu\textbackslash{}mathrm\{Fo\})\$.\%\% These four lines change the dcolumn to use text figures, instead of math figures.\%\% The reason is that some mitthesis font sets use different typefaces for text and math\%\% See: https://tex.stackexchange.com/a/376127/119566\textbackslash{}makeatletter \textbackslash{}newcolumntype\{T\}[3]\{>\{\textbackslash{}textfont0 =\textbackslash{}the\textbackslash{}font\textbackslash{}DC@\{\#1\}\{\#2\}\{\#3\}\}c<\{\textbackslash{}DC\textbackslash{}@end\}\} \% https://tex.stackexchange.com/a/376127/119566 \textbackslash{}newcolumntype\{j\}[1]\{T\{.\}\{.\}\{\#1\}\}\textbackslash{}makeatother\{\textbackslash{}footnotesize\% read documentation of longtable package for info on setting up a long table\% read documentation of array package and dcolumn package for info on column format specifiers\%\textbackslash{}newcolumntype\{X\}\{>\{\textbackslash{}hspace\{1ex\}\}c@\{\textbackslash{}hspace\{2ex\}\}c@\{\textbackslash{}hspace\{2ex\}\}c<\{\textbackslash{}hspace\{1ex\}\}\}\%longtable\{|||j\{3.2\}|X|X|X|||\} \% j column type is def'd several lines above. It is not a standard type, introduced to access the text (not math) font in this column.\% optional argument of \textbackslash{}caption suppresses citation hyperlink in list of figures by using \textbackslash{}CiteNolink rather than \textbackslash{}cite\textbackslash{}caption[One-term coefficients for one-dimensional heat conduction with a convective boundary condition. Data follow H. D. Baehr and K. Stephan\textasciitilde{}\textbackslash{}CiteNolink\{baehr1998\}.]\{One-term coefficients for one-dimensional heat conduction with a convective boundary condition. Data follow H. D. Baehr and K. Stephan\textasciitilde{}\textbackslash{}cite\{baehr1998\}.\}\%\textbackslash{}\textbackslash{}\textbackslash{}hline\textbackslash{}hline\textbackslash{}hline\& \textbackslash{}multicolumn\{3\}\{c|\}\{\textbackslash{}rule[0pt]\{0pt\}\{13pt\}\textbackslash{}textsf\{\textbackslash{}textit\{Plate\}\}\} \& \textbackslash{}multicolumn\{3\}\{c|\}\{\textbackslash{}textsf\{\textbackslash{}textit\{Cylinder\}\}\} \& \textbackslash{}multicolumn\{3\}\{c|||\}\{\textbackslash{}textsf\{\textbackslash{}textit\{Sphere\}\}\}\textbackslash{}\textbackslash{} \textbackslash{}cline\{2-10\} \textbackslash{}multicolumn\{1\}\{|||c|\}\{\textbackslash{}raisebox\{1.5ex\}[0cm][0cm]\{Bi\}\} \& \$\textbackslash{}lambda\_1\$\textbackslash{}rule[0pt]\{0pt\}\{11pt\} \& \$A\_1\$ \& \$D\_1\$ \& \$\textbackslash{}lambda\_1\$ \& \$A\_1\$ \& \$D\_1\$ \& \$\textbackslash{}lambda\_1\$ \& \$A\_1\$ \& \$D\_1\$ \textbackslash{}\textbackslash{} \textbackslash{}hline \textbackslash{}endfirsthead\textbackslash{}caption[]\{(continued)\} \textbackslash{}\textbackslash{}\textbackslash{}hline\textbackslash{}hline\textbackslash{}hline\& \textbackslash{}multicolumn\{3\}\{c|\}\{\textbackslash{}rule[0pt]\{0pt\}\{13pt\}\textbackslash{}textsf\{\textbackslash{}textit\{Plate\}\}\} \& \textbackslash{}multicolumn\{3\}\{c|\}\{\textbackslash{}textsf\{\textbackslash{}textit\{Cylinder\}\}\} \& \textbackslash{}multicolumn\{3\}\{c|||\}\{\textbackslash{}textsf\{\textbackslash{}textit\{Sphere\}\}\}\textbackslash{}\textbackslash{} \textbackslash{}cline\{2-10\}\textbackslash{}multicolumn\{1\}\{|||c|\}\{\textbackslash{}raisebox\{1.5ex\}[0cm][0cm]\{Bi\}\} \& \$\textbackslash{}lambda\_1\$\textbackslash{}rule[0pt]\{0pt\}\{11pt\} \& \$A\_1\$ \& \$D\_1\$ \& \$\textbackslash{}lambda\_1\$ \& \$A\_1\$ \& \$D\_1\$ \& \$\textbackslash{}lambda\_1\$ \& \$A\_1\$ \& \$D\_1\$ \textbackslash{}\textbackslash{} \textbackslash{}hline\&\&\&\&\&\&\&\&\&\textbackslash{}\textbackslash{}[-1ex]\textbackslash{}endhead\textbackslash{}hline\textbackslash{}hline\textbackslash{}hline\textbackslash{}endfoot\textbackslash{}hline\textbackslash{}hline\textbackslash{}hline\textbackslash{}endlastfoot 0.01 \& 0.09983 \& 1.0017 \& 1.0000 \& 0.14124 \& 1.0025 \& 1.0000 \& 0.17303 \& 1.0030 \& 1.0000\textbackslash{}rule[0pt]\{0pt\}\{15pt\} \textbackslash{}\textbackslash{} 0.02 \& 0.14095 \& 1.0033 \& 1.0000 \& 0.19950 \& 1.0050 \& 1.0000 \& 0.24446 \& 1.0060 \& 1.0000 \textbackslash{}\textbackslash{} 0.03 \& 0.17234 \& 1.0049 \& 1.0000 \& 0.24403 \& 1.0075 \& 1.0000 \& 0.29910 \& 1.0090 \& 1.0000 \textbackslash{}\textbackslash{} 0.04 \& 0.19868 \& 1.0066 \& 1.0000 \& 0.28143 \& 1.0099 \& 1.0000 \& 0.34503 \& 1.0120 \& 1.0000 \textbackslash{}\textbackslash{} 0.05 \& 0.22176 \& 1.0082 \& 0.9999 \& 0.31426 \& 1.0124 \& 0.9999 \& 0.38537 \& 1.0150 \& 1.0000 \textbackslash{}\textbackslash{} 0.06 \& 0.24253 \& 1.0098 \& 0.9999 \& 0.34383 \& 1.0148 \& 0.9999 \& 0.42173 \& 1.0179 \& 0.9999 \textbackslash{}\textbackslash{} 0.07 \& 0.26153 \& 1.0114 \& 0.9999 \& 0.37092 \& 1.0173 \& 0.9999 \& 0.45506 \& 1.0209 \& 0.9999 \textbackslash{}\textbackslash{} 0.08 \& 0.27913 \& 1.0130 \& 0.9999 \& 0.39603 \& 1.0197 \& 0.9999 \& 0.48600 \& 1.0239 \& 0.9999 \textbackslash{}\textbackslash{} 0.09 \& 0.29557 \& 1.0145 \& 0.9998 \& 0.41954 \& 1.0222 \& 0.9998 \& 0.51497 \& 1.0268 \& 0.9999 \textbackslash{}\textbackslash{} 0.10 \& 0.31105 \& 1.0161 \& 0.9998 \& 0.44168 \& 1.0246 \& 0.9998 \& 0.54228 \& 1.0298 \& 0.9998 \textbackslash{}\textbackslash{}[6pt] \%0.15 \& 0.37788 \& 1.0237 \& 0.9995 \& 0.53761 \& 1.0365 \& 0.9995 \& 0.66086 \& 1.0445 \& 0.9996 \textbackslash{}\textbackslash{}* 0.20 \& 0.43284 \& 1.0311 \& 0.9992 \& 0.61697 \& 1.0483 \& 0.9992 \& 0.75931 \& 1.0592 \& 0.9993 \textbackslash{}\textbackslash{} 0.25 \& 0.48009 \& 1.0382 \& 0.9988 \& 0.68559 \& 1.0598 \& 0.9988 \& 0.84473 \& 1.0737 \& 0.9990 \textbackslash{}\textbackslash{} 0.30 \& 0.52179 \& 1.0450 \& 0.9983 \& 0.74646 \& 1.0712 \& 0.9983 \& 0.92079 \& 1.0880 \& 0.9985 \textbackslash{}\textbackslash{} 0.40 \& 0.59324 \& 1.0580 \& 0.9971 \& 0.85158 \& 1.0931 \& 0.9970 \& 1.05279 \& 1.1164 \& 0.9974 \textbackslash{}\textbackslash{} 0.50 \& 0.65327 \& 1.0701 \& 0.9956 \& 0.94077 \& 1.1143 \& 0.9954 \& 1.16556 \& 1.1441 \& 0.9960 \textbackslash{}\textbackslash{} 0.60 \& 0.70507 \& 1.0814 \& 0.9940 \& 1.01844 \& 1.1345 \& 0.9936 \& 1.26440 \& 1.1713 \& 0.9944 \textbackslash{}\textbackslash{} 0.70 \& 0.75056 \& 1.0918 \& 0.9922 \& 1.08725 \& 1.1539 \& 0.9916 \& 1.35252 \& 1.1978 \& 0.9925 \textbackslash{}\textbackslash{} 0.80 \& 0.79103 \& 1.1016 \& 0.9903 \& 1.14897 \& 1.1724 \& 0.9893 \& 1.43203 \& 1.2236 \& 0.9904 \textbackslash{}\textbackslash{} 0.90 \& 0.82740 \& 1.1107 \& 0.9882 \& 1.20484 \& 1.1902 \& 0.9869 \& 1.50442 \& 1.2488 \& 0.9880 \textbackslash{}\textbackslash{}[6pt] \%1.00 \& 0.86033 \& 1.1191 \& 0.9861 \& 1.25578 \& 1.2071 \& 0.9843 \& 1.57080 \& 1.2732 \& 0.9855 \textbackslash{}\textbackslash{}* 1.10 \& 0.89035 \& 1.1270 \& 0.9839 \& 1.30251 \& 1.2232 \& 0.9815 \& 1.63199 \& 1.2970 \& 0.9828 \textbackslash{}\textbackslash{} 1.20 \& 0.91785 \& 1.1344 \& 0.9817 \& 1.34558 \& 1.2387 \& 0.9787 \& 1.68868 \& 1.3201 \& 0.9800 \textbackslash{}\textbackslash{} 1.30 \& 0.94316 \& 1.1412 \& 0.9794 \& 1.38543 \& 1.2533 \& 0.9757 \& 1.74140 \& 1.3424 \& 0.9770 \textbackslash{}\textbackslash{} 1.40 \& 0.96655 \& 1.1477 \& 0.9771 \& 1.42246 \& 1.2673 \& 0.9727 \& 1.79058 \& 1.3640 \& 0.9739 \textbackslash{}\textbackslash{} 1.50 \& 0.98824 \& 1.1537 \& 0.9748 \& 1.45695 \& 1.2807 \& 0.9696 \& 1.83660 \& 1.3850 \& 0.9707 \textbackslash{}\textbackslash{} 1.60 \& 1.00842 \& 1.1593 \& 0.9726 \& 1.48917 \& 1.2934 \& 0.9665 \& 1.87976 \& 1.4052 \& 0.9674 \textbackslash{}\textbackslash{} 1.70 \& 1.02725 \& 1.1645 \& 0.9703 \& 1.51936 \& 1.3055 \& 0.9633 \& 1.92035 \& 1.4247 \& 0.9640 \textbackslash{}\textbackslash{}* 1.80 \& 1.04486 \& 1.1695 \& 0.9680 \& 1.54769 \& 1.3170 \& 0.9601 \& 1.95857 \& 1.4436 \& 0.9605 \textbackslash{}\textbackslash{}* 1.90 \& 1.06136 \& 1.1741 \& 0.9658 \& 1.57434 \& 1.3279 \& 0.9569 \& 1.99465 \& 1.4618 \& 0.9570 \textbackslash{}\textbackslash{}[6pt] \%2.00 \& 1.07687 \& 1.1785 \& 0.9635 \& 1.59945 \& 1.3384 \& 0.9537 \& 2.02876 \& 1.4793 \& 0.9534 \textbackslash{}\textbackslash{}* 2.20 \& 1.10524 \& 1.1864 \& 0.9592 \& 1.64557 \& 1.3578 \& 0.9472 \& 2.09166 \& 1.5125 \& 0.9462 \textbackslash{}\textbackslash{} 2.40 \& 1.13056 \& 1.1934 \& 0.9549 \& 1.68691 \& 1.3754 \& 0.9408 \& 2.14834 \& 1.5433 \& 0.9389 \textbackslash{}\textbackslash{} 2.60 \& 1.15330 \& 1.1997 \& 0.9509 \& 1.72418 \& 1.3914 \& 0.9345 \& 2.19967 \& 1.5718 \& 0.9316 \textbackslash{}\textbackslash{} 2.80 \& 1.17383 \& 1.2052 \& 0.9469 \& 1.75794 \& 1.4059 \& 0.9284 \& 2.24633 \& 1.5982 \& 0.9243 \textbackslash{}\textbackslash{} 3.00 \& 1.19246 \& 1.2102 \& 0.9431 \& 1.78866 \& 1.4191 \& 0.9224 \& 2.28893 \& 1.6227 \& 0.9171 \textbackslash{}\textbackslash{} 3.50 \& 1.23227 \& 1.2206 \& 0.9343 \& 1.85449 \& 1.4473 \& 0.9081 \& 2.38064 \& 1.6761 \& 0.8995 \textbackslash{}\textbackslash{} 4.00 \& 1.26459 \& 1.2287 \& 0.9264 \& 1.90808 \& 1.4698 \& 0.8950 \& 2.45564 \& 1.7202 \& 0.8830 \textbackslash{}\textbackslash{}* 4.50 \& 1.29134 \& 1.2351 \& 0.9193 \& 1.95248 \& 1.4880 \& 0.8830 \& 2.51795 \& 1.7567 \& 0.8675 \textbackslash{}\textbackslash{}* 5.00 \& 1.31384 \& 1.2402 \& 0.9130 \& 1.98981 \& 1.5029 \& 0.8721 \& 2.57043 \& 1.7870 \& 0.8533 \textbackslash{}\textbackslash{}[6pt] \%6.00 \& 1.34955 \& 1.2479 \& 0.9021 \& 2.04901 \& 1.5253 \& 0.8532 \& 2.65366 \& 1.8338 \& 0.8281 \textbackslash{}\textbackslash{}* 7.00 \& 1.37662 \& 1.2532 \& 0.8932 \& 2.09373 \& 1.5411 \& 0.8375 \& 2.71646 \& 1.8673 \& 0.8069 \textbackslash{}\textbackslash{} 8.00 \& 1.39782 \& 1.2570 \& 0.8858 \& 2.12864 \& 1.5526 \& 0.8244 \& 2.76536 \& 1.8920 \& 0.7889 \textbackslash{}\textbackslash{} 9.00 \& 1.41487 \& 1.2598 \& 0.8796 \& 2.15661 \& 1.5611 \& 0.8133 \& 2.80443 \& 1.9106 \& 0.7737 \textbackslash{}\textbackslash{} 10.00 \& 1.42887 \& 1.2620 \& 0.8743 \& 2.17950 \& 1.5677 \& 0.8039 \& 2.83630 \& 1.9249 \& 0.7607 \textbackslash{}\textbackslash{} 12.00 \& 1.45050 \& 1.2650 \& 0.8658 \& 2.21468 \& 1.5769 \& 0.7887 \& 2.88509 \& 1.9450 \& 0.7397 \textbackslash{}\textbackslash{} 14.00 \& 1.46643 \& 1.2669 \& 0.8592 \& 2.24044 \& 1.5828 \& 0.7770 \& 2.92060 \& 1.9581 \& 0.7236 \textbackslash{}\textbackslash{} 16.00 \& 1.47864 \& 1.2683 \& 0.8541 \& 2.26008 \& 1.5869 \& 0.7678 \& 2.94756 \& 1.9670 \& 0.7109 \textbackslash{}\textbackslash{}* 18.00 \& 1.48830 \& 1.2692 \& 0.8499 \& 2.27556 \& 1.5898 \& 0.7603 \& 2.96871 \& 1.9734 \& 0.7007 \textbackslash{}\textbackslash{}* 20.00 \& 1.49613 \& 1.2699 \& 0.8464 \& 2.28805 \& 1.5919 \& 0.7542 \& 2.98572 \& 1.9781 \& 0.6922 \textbackslash{}\textbackslash{}[6pt] \%25.00 \& 1.51045 \& 1.2710 \& 0.8400 \& 2.31080 \& 1.5954 \& 0.7427 \& 3.01656 \& 1.9856 \& 0.6766 \textbackslash{}\textbackslash{}* 30.00 \& 1.52017 \& 1.2717 \& 0.8355 \& 2.32614 \& 1.5973 \& 0.7348 \& 3.03724 \& 1.9898 \& 0.6658 \textbackslash{}\textbackslash{} 35.00 \& 1.52719 \& 1.2721 \& 0.8322 \& 2.33719 \& 1.5985 \& 0.7290 \& 3.05207 \& 1.9924 \& 0.6579 \textbackslash{}\textbackslash{} 40.00 \& 1.53250 \& 1.2723 \& 0.8296 \& 2.34552 \& 1.5993 \& 0.7246 \& 3.06321 \& 1.9942 \& 0.6519 \textbackslash{}\textbackslash{} 50.00 \& 1.54001 \& 1.2727 \& 0.8260 \& 2.35724 \& 1.6002 \& 0.7183 \& 3.07884 \& 1.9962 \& 0.6434 \textbackslash{}\textbackslash{} 60.00 \& 1.54505 \& 1.2728 \& 0.8235 \& 2.36510 \& 1.6007 \& 0.7140 \& 3.08928 \& 1.9974 \& 0.6376 \textbackslash{}\textbackslash{} 80.00 \& 1.55141 \& 1.2730 \& 0.8204 \& 2.37496 \& 1.6013 \& 0.7085 \& 3.10234 \& 1.9985 \& 0.6303 \textbackslash{}\textbackslash{} 100.00 \& 1.55525 \& 1.2731 \& 0.8185 \& 2.38090 \& 1.6015 \& 0.7052 \& 3.11019 \& 1.9990 \& 0.6259 \textbackslash{}\textbackslash{}* 200.00 \& 1.56298 \& 1.2732 \& 0.8146 \& 2.39283 \& 1.6019 \& 0.6985 \& 3.12589 \& 1.9998 \& 0.6170 \textbackslash{}\textbackslash{}* \textbackslash{}infty \& 1.57080 \& 1.2732 \& 0.8106 \& 2.40483 \& 1.6020 \& 0.6917 \& 3.14159 \& 2.0000 \& 0.6079 \textbackslash{}\textbackslash{}[3pt] longtable\}\% longtable example\%\%\% Bibliography (biblatex) \%\%\%\%\%\%\%\%\%\%\%\%\%\%\%\%\%\%\%\%\%\%\%\%\%\%\%\%\%\%\%\%\%\%\%\%\%\%\%\%\%\%\%\%\%\%\%\%\%\%\%\%\%\%\%\%\%\%\%\%\%\%\%\%\%\%\%\%\%\%\{\textbackslash{}chapter\{\#1\}\textbackslash{}addcontentsline\{toc\}\{backmatter\}\{\textbackslash{}refname\}\} \% this sets the title of contents name for bibliography to \textbackslash{}refname (= References)\% change "backmatter" to "chapter" if you prefer a bold face entry in the table of contents\textbackslash{}printbibliography[title=\{\textbackslash{}refname\},heading=bibintoc]\% biblatex also supports chapter-by-chapter bibliography, https://tex.stackexchange.com/a/296502/119566\% see the biblatex manual, section 3.14.3document
\end{document}
