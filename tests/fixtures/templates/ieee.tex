\documentclass[conference]{IEEEtran}
\usepackage{cite}
\usepackage{amsmath,amssymb,amsfonts}
\usepackage{algorithmic}
\usepackage{graphicx}
\usepackage{textcomp}
\usepackage{xcolor}
\usepackage{caption}
\usepackage{newcomputermodern}
\makeatletter
\renewcommand\section{\@startsection{section}{1}{\z@}{0.5em}{0.3em}{\centering\normalfont\bfseries\MakeUppercase}}
\renewcommand\subsection{\@startsection{subsection}{2}{\z@}{0.5em}{0.3em}{\normalfont\bfseries}}
\renewcommand\subsubsection{\@startsection{subsubsection}{3}{\z@}{0pt}{0pt}{\normalfont\bfseries}}
\makeatother
\providecommand{\textsubscript}[1]{$_{\text{#1}}$}
\begin{document}
\setlength{\parindent}{1em}
\makeatletter
\twocolumn[{\begin{@twocolumnfalse}
\begin{center}
{\fontsize{24pt}{28pt}\selectfont\bfseries A Typesetting System to Untangle the Scientific Writing Process}
\vspace{1em}
\begin{tabular}{c@{\hspace{1.5em}}c}
\begin{tabular}{@{}c@{}}{\fontsize{11pt}{13.2pt}\selectfont First Author} \\ {\fontsize{9pt}{10.8pt}\selectfont\textit{Department of Computer Science}} \\ {\fontsize{9pt}{10.8pt}\selectfont University Name} \\ {\fontsize{9pt}{10.8pt}\selectfont City, Country} \\ {\fontsize{9pt}{10.8pt}\selectfont author@university.edu}\end{tabular} & \begin{tabular}{@{}c@{}}{\fontsize{11pt}{13.2pt}\selectfont Second Author} \\ {\fontsize{9pt}{10.8pt}\selectfont\textit{Department of Engineering}} \\ {\fontsize{9pt}{10.8pt}\selectfont Another University} \\ {\fontsize{9pt}{10.8pt}\selectfont City, Country} \\ {\fontsize{9pt}{10.8pt}\selectfont coauthor@university.edu}\end{tabular} \\
\end{tabular}
\vspace{1.5em}
\begin{minipage}{\linewidth}
\raggedright
\textbf{Abstract---}\textit{The process of scientific writing is often tangled up with the intricacies of typesetting, leading to frustration and wasted time for researchers. In this paper, we introduce Typst, a new typesetting system designed specifically for scientific writing. Typst untangles the typesetting process, allowing researchers to compose papers faster. In a series of experiments we demonstrate that Typst offers several advantages, including faster document creation, simplified syntax, and increased ease-of-use.}
\vspace{0.5em}
\textbf{Index Terms---}Scientific writing, Typesetting, Document creation, Syntax
\end{minipage}
\end{center}
\end{@twocolumnfalse}}]
\makeatother
\section{Introduction}

The introduction establishes the context and motivation for your research. Explain the problem you're addressing and why it matters.

In this paper, we present our approach to solving this problem. Our key contributions are:

\begin{itemize}
  \item First contribution
  \item Second contribution
  \item Third contribution
\end{itemize}

The rest of this paper is organized as follows. Section II discusses related work. Section III describes our methodology. Section IV presents our results, and Section V concludes.

\section{Related Work}

Discuss prior research and how your work relates to and differs from existing approaches.

Previous work in this area has focused on various aspects of the problem. Smith et al. proposed an approach based on... [1].

\section{Methodology}

Describe your approach, methods, algorithms, or system architecture in detail.

\subsection{System Overview}

Our system consists of three main components...

\subsection{Algorithm Design}

The core algorithm works as follows...

\[
E = m c^2
\]

\section{Experimental Results}

Present your experimental setup, datasets, and results.

\subsection{Experimental Setup}

We conducted experiments on...

\subsection{Results}

\begin{center}
\begin{tabular}{cccc}
\textbf{Method} & \textbf{Precision} & \textbf{Recall} & \textbf{F1} \\
Baseline & 0.72 & 0.68 & 0.70 \\
Ours & \textbf{0.89} & \textbf{0.85} & \textbf{0.87} \\
\end{tabular}
\captionof{table}{Comparison of methods on the benchmark dataset.}
\end{center}

Our method outperforms the baseline by 17\% on F1 score.

\section{Conclusion}

Summarize your contributions and suggest directions for future work.

In this paper, we presented... Future work includes...

\section*{References}
{\fontsize{9pt}{10.8pt}\selectfont
\begin{thebibliography}{99}
\bibitem{ref1} J. Smith, "Example paper title," in \textit{Proc. IEEE Conference}, 2023, pp. 1-10.
\bibitem{ref2} A. Jones and B. Williams, "Another example paper," \textit{IEEE Trans. on Computing}, vol. 42, no. 3, pp. 123-135, 2024.
\end{thebibliography}
}

\end{document}
