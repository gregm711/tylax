\documentclass{article}
\usepackage[margin=2cm]{geometry}
\usepackage{amsmath,amssymb}
\usepackage{graphicx}
\usepackage{hyperref}
\usepackage[table]{xcolor}
\usepackage{enumitem}
\usepackage{multirow}
\usepackage{multicol}
\usepackage{array}
\usepackage{textcomp}
\usepackage{float}
\usepackage{newcomputermodern}
\usepackage{setspace}
\setstretch{1.38}
\definecolor{accent}{HTML}{F59E0B}
\definecolor{light-bg}{HTML}{F8FAFC}
\definecolor{primary}{HTML}{1E40AF}
\definecolor{secondary}{HTML}{3B82F6}
\providecommand{\textsubscript}[1]{$_{\text{#1}}$}
\begin{document}

 \begingroup
\setlength{\tabcolsep}{2cm}
\setlength{\extrarowheight}{2cm}
\begin{tabular}{ccc}
\end{tabular}
\endgroup 

\par\vspace{1.5cm}

\begingroup
\setlength{\tabcolsep}{1.5cm}
\setlength{\extrarowheight}{1.5cm}
\begin{tabular}{ccc}
Introduction This research addresses the important problem of [topic]. Understanding this is crucial because:\\ First reason why this matters Second reason for importance Potential impact on the field\\ \textbf{Research Question:} How can we effectively [main question]?
\par\vspace{1cm}
Background Previous approaches have limitations:\\ \textbf{Method A} (Smith, 2023) Limited by assumption X Poor scalability\\ \textbf{Method B} (Jones, 2024) Requires expensive resources Not generalizable\\ Our approach addresses these gaps by...
\par\vspace{1cm}
Methods \textbf{Our Framework:}\\ {\footnotesize Input → Process A → Process B → Output } \\ \textbf{Key Innovation:}\\ We introduce a novel approach that combines [technique 1] with [technique 2] to achieve [benefit]. & Key Results "\#d1d5db" \textcolor{white}{\textbf{Method}} \textcolor{white}{\textbf{Acc.}} \textcolor{white}{\textbf{F1}} \textcolor{white}{\textbf{Time}} Baseline 1 72.3\% 0.70 45s Baseline 2 75.1\% 0.73 38s \textbf{Ours} \textbf{89.2\%} \textbf{0.87} \textbf{12s} \\ "\#d1fae5" {\large \textcolor[HTML]{065F46}{\textbf{ +14\% accuracy improvement \\ 3× faster than baselines }}}
\par\vspace{1cm}
Analysis \textbf{Why does our method work better?}\\ \textbf{Reason 1:} Explanation of first key factor contributing to improvements.\\ \textbf{Reason 2:} Explanation of second key factor.\\ \textbf{Reason 3:} Explanation of third key factor.\\ \textbf{Ablation Study:} Removing component X decreases performance by 8\%, confirming its importance. & Discussion \textbf{Implications:}\\ Finding 1 suggests that... Finding 2 indicates potential for... These results open new directions for...\\ \textbf{Limitations:}\\ Current approach assumes... May not generalize to... Requires further validation on...
\par\vspace{1cm}
Conclusion \textbf{Key Contributions:}\\ ✓ Contribution 1: Novel method for...\\ ✓ Contribution 2: Comprehensive evaluation...\\ ✓ Contribution 3: Open-source implementation... \\ \textbf{Future Work:} Extension to [new domain], integration with [other methods].
\par\vspace{1cm}
References {\scriptsize [1] Smith et al. "Paper title." \textit{Conference}, 2023.\\ [2] Jones \& Lee. "Another paper." \textit{Journal}, 2024.\\ [3] Chen et al. "Third reference." \textit{Venue}, 2024. }
\par\vspace{1cm}
 \begin{center}
  \begingroup
\setlength{\tabcolsep}{1cm}
\setlength{\extrarowheight}{1cm}
\begin{tabular}{cc}
{\footnotesize \textbf{Scan for:}} \begin{itemize}
  \item Paper PDF
  \item Code \& Data
  \item Demo
\end{itemize} \\
\end{tabular}
\endgroup  
\end{center}  \\
\end{tabular}
\endgroup
\end{document}
