\documentclass[12pt,a4paper]{article}
\usepackage{amsmath,amssymb}
\usepackage{graphicx}
\usepackage{hyperref}
\usepackage[table]{xcolor}
\usepackage{booktabs}
\usepackage{enumitem}
\usepackage{multirow}
\usepackage{multicol}
\usepackage{array}
\usepackage{textcomp}
\usepackage{float}
\usepackage{titlesec}
\titlespacing*{\section}{0pt}{1em}{0.5em}
\titlespacing*{\subsection}{0pt}{0.8em}{0.3em}
\titlespacing*{\subsubsection}{0pt}{0.6em}{0.2em}
\titlespacing*{\paragraph}{0pt}{0.5em}{0.1em}
\numberwithin{equation}{section}
\providecommand{\textsubscript}[1]{$_{\text{#1}}$}
\begin{document}

\begin{center}
{\LARGE \textbf{How to Write Theses With Two Line Titles}}
\par\vspace{1em}
Your name
\par\vspace{0.5em}

\end{center}

\textbf{Thesis Metadata}
\par\vspace{0.5em}
 \begin{itemize}
  \item \textbf{Department:} Your department
  \item \textbf{Advisor:} Professor name 1
  \item \textbf{Reader:} Professor name 2
  \item \textbf{Reader:} Professor name 3
\end{itemize}

\section{Abstract}

My abstract.

\section{Acknowledgements}

I would like to thank...

\section{Introduction}

\subsection{Background}

Broad overview.

\subsubsection{Specific background}

Specific text.

\section{Long paper 1 chapter title}

Short chapter title

\subsection{Abstract}

Concise introduction, motivation, results, and conclusions.

\subsection{Introduction}

Examples of citations: we knew X from \cite{Croote201616022} and Y from \cite{Croote20181306}.

\subsection{Results}

\subsubsection{Result 1}

Here is a reference to Figure \cite{fig-paper1-fig1}. We found bacteria, roughly 5 mu m in length, that live at 95degree C for roughly 90\% of the year. More symbols: $, #, 10#super[5],$alpha $,$beta $,$gamma $,$kappa

\begin{equation}
, *bold*, _italics_.
Quotation marks are "correctly oriented" thanks to the csquotes package.
Now an inline equation:
\end{equation}

E = m c\textasciicircum{}(2)$. Now a reference to Equation @eqn-paper1-eqn1:$ d\textit{(t) = frac(c - p n}(t), n\textit{(t)) $<eqn-paper1-eqn1> #figure( image("figs/paper1/fig1.png") ) <fig-paper1-fig1> There will be more text here. More text here. More text here. More text here. More text here. More text here. === Result 2 We discovered something else. Here is a reference to Table @tab-paper1-tab1. #figure()[ #table( columns: (auto), align: (left), [c|c|c|c *Col 1*], [*Col 2*], [*Col 3*], [*Col 4*], table.hline(), [Text \in Row 1a], [Text \in Row 1b], [Text \in Row 1c], [Lots of text \in Row 1d], [], [Row 1.5b], [Row 1.5c], [Row 1.5d], table.hline(), [Row 2a], [Row 2b], [Row 2c], [Row 2d], table.hline(), [Row 3a], [Row 3b], [Row 3c], [Row 3d], ) ] <tab-paper1-tab1> More text here. More text here. More text here. More text here. FixAwkwardSpacingWithSloppypar FixAwkwardSpacingWithSloppypar FixAwkwardSpacingWithSloppypar FixAwkwardSpacingWithSloppypar FixAwkwardSpacingWithSloppypar. More text here. More text here. == Conclusions The end of the paper. = Concluding Remarks I conclude\ldots #bibliography("mybib.typst.bib")$ }
\end{document}
