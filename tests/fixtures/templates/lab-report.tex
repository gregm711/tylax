\documentclass[11pt,letterpaper]{article}
\usepackage[margin=1in]{geometry}
\usepackage{amsmath,amssymb}
\usepackage{graphicx}
\usepackage{hyperref}
\usepackage[table]{xcolor}
\usepackage{enumitem}
\usepackage{multirow}
\usepackage{multicol}
\usepackage{array}
\usepackage{textcomp}
\usepackage{newcomputermodern}
\usepackage{setspace}
\setstretch{1.81}
\providecommand{\textsubscript}[1]{$_{\text{#1}}$}
\begin{document}

\begin{center}
\\\textbf{ Laboratory Report: \\ Determination of the Acceleration Due to Gravity \\ Using a Simple Pendulum }\\ Course: PHYS 101 - Introduction to Physics \\ Lab Section: Tuesday 2:00 PM \\ \textbf{Submitted by:} \\ Your Name \\ Student ID: 123456789 \\ \textbf{Lab Partner(s):} \\ Partner Name \\ \textbf{Instructor:} \\ Dr. Professor Name \\ \textbf{Date Performed:} November 15, 2024 \\ \textbf{Date Submitted:} November 22, 2024
\end{center}

\section{Abstract}

This experiment investigated the relationship between the period of a simple pendulum and the acceleration due to gravity. By measuring the period of oscillation for various pendulum lengths, we determined the local gravitational acceleration to be

\begin{equation}
g = 9.72 plus.minus 0.15 "m/s"^2
\end{equation}

. This value agrees with the accepted value of

\begin{equation}
9.81 "m/s"^2
\end{equation}

within experimental uncertainty. The primary sources of error include timing precision and small-angle approximation limitations.

\section{Introduction}

\subsection{Background}

The simple pendulum is a classic system in physics that demonstrates periodic motion. For small angular displacements, the period

\begin{equation}
T
\end{equation}

of a simple pendulum is given by:

\begin{equation}
T = 2 \pi sqrt(L / g)
\label{eq:period}
\end{equation}

where

\begin{equation}
L
\end{equation}

is the length of the pendulum and

\begin{equation}
g
\end{equation}

is the acceleration due to gravity.

\subsection{Objectives}

The objectives of this experiment were to:

\begin{enumerate}
  \item Verify the relationship between pendulum length and period
  \item Determine the local value of gravitational acceleration
  \item Analyze sources of experimental uncertainty
\end{enumerate}

\subsection{Hypothesis}

Based on \eqref{eq:period}, we hypothesize that the period squared (

\begin{equation}
T^2
\end{equation}

) will be linearly proportional to the pendulum length (

\begin{equation}
L
\end{equation}

), with the slope related to the gravitational acceleration.

\section{Materials and Methods}

\subsection{Apparatus}

The following equipment was used:

\begin{itemize}
  \item Metal bob (mass: 50 g)
  \item String (low stretch, length: 1.5 m)
  \item Meter stick (precision: 1 mm)
  \item Digital stopwatch (precision: 0.01 s)
  \item Protractor
  \item Ring stand and clamp
\end{itemize}

\subsection{Procedure}

\begin{enumerate}
  \item The pendulum was assembled by attaching the metal bob to the string and securing it to the ring stand.
  \item The length $L$ was measured from the pivot point to the center of mass of the bob.
  \item The pendulum was displaced by approximately 10° from vertical.
  \item The time for 10 complete oscillations was measured three times for each length.
  \item Steps 2-4 were repeated for lengths of 0.30, 0.50, 0.70, 0.90, and 1.10 m.
\end{enumerate}

\begin{table}[h]
\centering
\begingroup
\setlength{\tabcolsep}{8pt}
\setlength{\arrayrulewidth}{0.5pt}
\begin{tabular}{|c|c|c|c|c|}
\hline
\textbf{Length (m)} & \textbf{Trial 1 (s)} & \textbf{Trial 2 (s)} & \textbf{Trial 3 (s)} & \textbf{Mean (s)} \\
\hline
0.30 & 10.92 & 10.98 & 10.95 & 10.95 \\
\hline
0.50 & 14.18 & 14.22 & 14.15 & 14.18 \\
\hline
0.70 & 16.75 & 16.82 & 16.78 & 16.78 \\
\hline
0.90 & 19.02 & 18.98 & 19.05 & 19.02 \\
\hline
1.10 & 21.05 & 21.12 & 21.08 & 21.08 \\
\hline
\end{tabular}
\endgroup
\caption{Time measurements for 10 oscillations at various pendulum lengths.}
\label{tab:raw-data}
\end{table}

\section{Results}

\subsection{Data Processing}

The period for each length was calculated by dividing the mean time by 10. The results are summarized in \ref{tab:processed}.

\begin{table}[h]
\centering
\begingroup
\setlength{\tabcolsep}{8pt}
\setlength{\arrayrulewidth}{0.5pt}
\begin{tabular}{|c|c|c|c|}
\hline
\textbf{$L$ (m)} & \textbf{$T$ (s)} & \textbf{$T^2$ (s²)} & \textbf{$\delta T^2$ (s²)} \\
\hline
0.30 & 1.095 & 1.199 & 0.022 \\
\hline
0.50 & 1.418 & 2.011 & 0.028 \\
\hline
0.70 & 1.678 & 2.816 & 0.034 \\
\hline
0.90 & 1.902 & 3.618 & 0.038 \\
\hline
1.10 & 2.108 & 4.444 & 0.042 \\
\hline
\end{tabular}
\endgroup
\caption{Processed data showing period and period squared.}
\label{tab:processed}
\end{table}

\subsection{Analysis}

From \eqref{eq:period}, we can derive:

\begin{equation}
T^2 = (4 \pi^2) / g L
\label{eq:linear}
\end{equation}

This predicts a linear relationship between

\begin{equation}
T^2
\end{equation}

and

\begin{equation}
L
\end{equation}

with slope

\begin{equation}
m = 4 \pi^2 / g
\end{equation}

.

Performing a linear regression on our data yields:

\begin{itemize}
  \item Slope: $m = 4.06 plus.minus 0.12 "s"^2"/""m"$
  \item Y-intercept: $b = -0.02 plus.minus 0.08 "s"^2$
  \item Correlation coefficient: $R^2 = 0.998$
\end{itemize}

\subsection{Calculated Gravitational Acceleration}

Solving for

\begin{equation}
g
\end{equation}

:

\begin{equation}
g = (4 \pi^2) / m = (4 \pi^2) / 4.06 = 9.72 "m/s"^2
\end{equation}

The uncertainty in

\begin{equation}
g
\end{equation}

is calculated using error propagation:

\begin{equation}
\delta g = g (\delta m) / m = 9.72 times (0.12 / 4.06) = 0.29 "m/s"^2
\end{equation}

Therefore, our experimental value is:

\begin{equation}
g = 9.72 plus.minus 0.29 "m/s"^2
\end{equation}

\section{Discussion}

\subsection{Comparison with Accepted Value}

The accepted value of gravitational acceleration at sea level is

\begin{equation}
g = 9.81 "m/s"^2
\end{equation}

. Our experimental value of

\begin{equation}
9.72 plus.minus 0.29 "m/s"^2
\end{equation}

agrees with the accepted value within experimental uncertainty.

The percent error is:

\begin{equation}
"Percent Error" = (|9.81 - 9.72|) / 9.81 times 100% = 0.92%
\end{equation}

\subsection{Sources of Error}

Several factors contributed to experimental uncertainty:

\textbf{Systematic Errors:}

\begin{itemize}
  \item The small-angle approximation ($sin \theta \approx \theta$) is only valid for angles less than \nobreakspace{}15°
  \item Air resistance was neglected
  \item The string was assumed to be massless
\end{itemize}

\textbf{Random Errors:}

\begin{itemize}
  \item Reaction time in starting/stopping the stopwatch
  \item Determining the exact endpoint of oscillations
  \item Slight variations in release angle
\end{itemize}

\subsection{Improvements}

Future experiments could be improved by:

\begin{enumerate}
  \item Using a photogate timer to reduce timing uncertainty
  \item Conducting measurements in a vacuum chamber to eliminate air resistance
  \item Using smaller initial displacements to better satisfy the small-angle approximation
\end{enumerate}

\section{Conclusion}

This experiment successfully determined the acceleration due to gravity using a simple pendulum. The measured value of

\begin{equation}
g = 9.72 plus.minus 0.29 "m/s"^2
\end{equation}

agrees with the accepted value of

\begin{equation}
9.81 "m/s"^2
\end{equation}

within experimental uncertainty (0.92\% error). The linear relationship between

\begin{equation}
T^2
\end{equation}

and

\begin{equation}
L
\end{equation}

was confirmed with a correlation coefficient of

\begin{equation}
R^2 = 0.998
\end{equation}

, supporting the theoretical model.

\section{References}

[1] Halliday, D., Resnick, R., and Walker, J. \textit{Fundamentals of Physics}, 11th ed. Wiley, 2018.\\[2] Taylor, J. R. \textit{An Introduction to Error Analysis}, 2nd ed. University Science Books, 1997.\\[3] PHYS 101 Laboratory Manual, Fall 2024.

\section{Appendix: Sample Calculations}

\textbf{Calculation of Period:}

\begin{equation}
T = (10.95 "s") / 10 = 1.095 "s"
\end{equation}

\textbf{Calculation of $T^2$:}

\begin{equation}
T^2 = (1.095)^2 = 1.199 "s"^2
\end{equation}

\textbf{Uncertainty in $T^2$:}

\begin{equation}
\delta(T^2) = 2 T dot \delta T = 2(1.095)(0.01) = 0.022 "s"^2
\end{equation}

\textbf{Slope from Linear Regression:}

\begin{equation}
m = (n \sum x_i y_i - \sum x_i \sum y_i) / (n \sum x_i^2 - (\sum x_i)^2)
\end{equation}
\end{document}
