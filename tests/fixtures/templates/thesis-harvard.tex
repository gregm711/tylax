\documentclass[11pt,a4paper]{article}
\usepackage{amsmath,amssymb}
\usepackage{graphicx}
\usepackage{hyperref}
\usepackage[table]{xcolor}
\usepackage{booktabs}
\usepackage{enumitem}
\usepackage{multirow}
\usepackage{multicol}
\usepackage{array}
\usepackage{textcomp}
\usepackage{float}
\usepackage{titlesec}
\titlespacing*{\section}{0pt}{1em}{0.5em}
\titlespacing*{\subsection}{0pt}{0.8em}{0.3em}
\titlespacing*{\subsubsection}{0pt}{0.6em}{0.2em}
\titlespacing*{\paragraph}{0pt}{0.5em}{0.1em}
\providecommand{\textsubscript}[1]{$_{\text{#1}}$}
\begin{document}

\begin{center}
{\LARGE \textbf{Essays on Thesis-formatting}}
\par\vspace{1em}
{\normalsize Econ Gradstudent}
\par\vspace{0.5em}

\end{center}

 \begin{center}
\textbf{Abstract}
\end{center} An abstract should be less than 350 words. Here's some filler text.

\section{Acknowledgments}

\section{Dedication}

To my parents

\section{Introduction}

\label{ch-intro} Introductiory chapter that talks about all three papers for a little bit longer than the abstract.

\section{Hook\footnote{Co-authored with my advisor}}
\label{ch-1}

\subsection{Introduction}
\label{section-intro}

Block Quotations (quotation and quote environments) are supposed to be single-spaced with each entry, and double-spaced between. The class file does this automatically. For example:

\begin{quote}
Dummy quote.
\end{quote}

\begin{quote}
Dummy quotation.
\end{quote}

\subsection{Motivating Example}

Table \cite{tab-label} shows stuff.

\begin{table}[H]
\centering
\begin{tabular}{|l|l|l|}
\hline
Tables & should & Be \\
\hline
double & spaced & unless \\
\hline
 & they are & long \\
\hline
This & table & is \\
\hline
getting & long & so \\
\hline
I & manually & set \\
\hline
it & to & single \\
\hline
spacing using \\
\hline
\end{tabular}
\label{tab-label}
\end{table}

Table \cite{tab-label2} shows stuff also.

\begin{table}[H]
\centering
\begin{tabular}{|l|l|l|l|l|}
\hline
Table & should & be & placed & within \\
\hline
text, & as & close & to & its first mention \\
\hline
as & possible. & Not at the end & of a chapter & or dissertation \\
\hline
\end{tabular}
\caption{Use consistent format for captions}
\label{tab-label2}
\end{table}

\section{Line\footnote{Co-authored with my other advisor}}
\label{ch-2}

\subsection{Introduction}

\subsection{Potential outcomes framework}
\label{sec-potent-outc-fram}

\footnote{Footnotes are single-spaced. }\footnote{Space between foonotes is doublespaced. }

\subsection{Conclusion}

I conclude that:

\begin{itemize}
  \item Lorem ipsum
  \item Dolor sit amet
  \item Consectetur
\end{itemize}

\section{Sinker}
\label{ch-3}

\subsection{Introduction}

Some people just cite papers in introductions for no reason. \cite{ar49} \cite{pearson01} \cite{spe04}.

\subsection{Setup}
\label{sec-potent-outc-fram}

See Figure \cite{fig-figure1} for illustration.

\begin{figure}[H]
\centering

\caption{Captions for figures go at the bottom of the figure.}
\label{fig-figure1}
\end{figure}

\subsection{Conclusion}

\bibliographystyle{plain}
\bibliography{references.typst}

\section{Appendix to Chapter \cite{ch-1}}
\label{cha-append-chapt-refch-1}

\subsection{Auxiliary Lemmata}

Fundamental identity

\[
e^(i \pi)= - 1 .
\]

Equivalence relation

\[
A = B .
\]

\subsection{Proofs}

\section{Appendix to Chapter \cite{ch-3}}

\subsection{Proofs}

\subsection{Supplementary Tables and Figures}

\begin{tabular}{|l|}
\hline
cc\\ []Optional Short caption (used in list of tables)A long table \label{grid-mlmmh} \\
\hline
Heading that appears \\
\hline
on first page only \\
\hline
 \\
\hline
[](continued) \\
\hline
Heading that appears \\
\hline
on all pages \\
\hline
 \\
\hline
\multicolumn{2}{|l|}{Continued on next page} \\
\hline
Contrary to popular \\
\hline
belief, Lorem Ipsum \\
\hline
is \\
\hline
not \\
\hline
simply \\
\hline
random \\
\hline
text \\
\hline
. It \\
\hline
has \\
\hline
roots \\
\hline
in \\
\hline
a \\
\hline
piece \\
\hline
of \\
\hline
classical \\
\hline
Latin \\
\hline
literature \\
\hline
from \\
\hline
45 \\
\hline
BC \\
\hline
, making \\
\hline
it \\
\hline
over \\
\hline
2000 \\
\hline
years old. Richard \\
\hline
Mc \\
\hline
Clintock \\
\hline
, a \\
\hline
Latin \\
\hline
professor \\
\hline
at \\
\hline
Hampden \\
\hline
-Sydney \\
\hline
College \\
\hline
in \\
\hline
Virginia \\
\hline
, looked \\
\hline
up \\
\hline
one \\
\hline
of \\
\hline
the \\
\hline
more \\
\hline
obscure \\
\hline
Latin \\
\hline
words \\
\hline
, consectetur \\
\hline
, from \\
\hline
a \\
\hline
Lorem \\
\hline
Ipsum \\
\hline
passage \\
\hline
, and \\
\hline
going \\
\hline
through \\
\hline
the \\
\hline
cites \\
\hline
of \\
\hline
the word in \\
\hline
classical \\
\hline
literature , discovered the \\
\hline
undoubtable \\
\hline
source. Lorem Ipsum \\
\hline
comes \\
\hline
from \\
\hline
sections \\
\hline
1 \\
\hline
.10 \\
\hline
.32 \\
\hline
and \\
\hline
1 \\
\hline
.10 \\
\hline
.33 \\
\hline
of \\
\hline
"de \\
\hline
Finibus \\
\hline
Bonorum \\
\hline
et \\
\hline
Malorum \\
\hline
" (The \\
\hline
Extremes \\
\hline
of \\
\hline
Good \\
\hline
and \\
\hline
Evil \\
\hline
) by \\
\hline
Cicero \\
\hline
, written \\
\hline
in \\
\hline
45 \\
\hline
BC \\
\hline
. This \\
\hline
book \\
\hline
is \\
\hline
a \\
\hline
treatise \\
\hline
on \\
\hline
the \\
\hline
theory \\
\hline
of \\
\hline
ethics \\
\hline
, very \\
\hline
popular \\
\hline
during \\
\hline
the \\
\hline
Renaissance \\
\hline
. The \\
\hline
first \\
\hline
line \\
\hline
of \\
\hline
Lorem \\
\hline
Ipsum, "Lorem ipsum \\
\hline
dolor \\
\hline
sit \\
\hline
amet \\
\hline
..", comes from a \\
\hline
line \\
\hline
in \\
\hline
section 1.10.32. \\
\hline
\end{tabular}

\begin{figure}[H]
\centering

\caption{Supplementary Figure}
\label{fig-figuresup1}
\end{figure}

\begin{figure}[H]
\centering

\caption{Another Figure}
\label{fig-figuresup3}
\end{figure}
\end{document}
